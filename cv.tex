% This work may be distributed and/or modified under the
% conditions of the LaTeX Project Public License version 1.3c,
% available at http://www.latex-project.org/lppl/.


\documentclass[12pt,a4paper]{moderncv}

% moderncv themes
\moderncvtheme[grey]{classic}                 % optional argument are 'blue' (default), 'orange', 'red', 'green', 'grey' and 'roman' (for roman fonts, instead of sans serif fonts)

% character encoding
\usepackage[utf8]{inputenc}                   % replace by the encoding you are using

% adjust the page margins
\usepackage[scale=0.8]{geometry}

\newcommand*{\cventryline}[1]{\cvitem{}{\small#1}}

% personal data
\firstname{Ignacio}
\familyname{Casal Quinteiro}
\title{Computer engineer}
\photo[90pt][0pt]{picture}


%----------------------------------------------------------------------------------
%            content
%----------------------------------------------------------------------------------
\begin{document}
\maketitle

\section{Personal information}
\cvdoubleitem{Date of birth}{11th July 1986}{Mobile}{(0034) 654 43 90 69}
\cvdoubleitem{Address}{Fuente del oeste 57}{Email}{\href{mailto:icquinteiro@gmail.com}{icquinteiro@gmail.com}}
\cvdoubleitem{}{36900, Marín (Spain)}{LinkedIn}{\href{https://www.linkedin.com/in/ignacio-casal-quinteiro-11622635}{https://goo.gl/QBk208}}
\cvdoubleitem{Nationality}{Spanish}{Blog}{\href{https://blogs.gnome.org/nacho}{https://goo.gl/Qw9zCl}}
\cvdoubleitem{Status}{Married}{OpenHub}{\href{https://www.openhub.net/accounts/nacho}{https://goo.gl/wkzlIm}}

\section{Summary}
\cvline{}{As a highly-motivated and experienced \textbf{Software Developer}, I have a proven track on a broad range of \textbf{Open Source} projects, as well as on the field of \textbf{Virtualization} and \textbf{HPC} technologies.}
\cvline{}{During the past years I had the opportunity to work with people from all around the world \textbf{writing and reviewing} code using \textbf{C}, \textbf{C++}, \textbf{Python}, \textbf{Javascript} or \textbf{Bash}. In order to build those projects \textbf{Autotools} and \textbf{Visual Studio} have been a must and as version control system I have a lot of experience with \textbf{Git} as well as \textbf{Subversion}.}
\cvline{}{Working for NICE, has also given me a lot of knowledge on \textbf{Virtualization}, \textbf{HPC}, \textbf{Network programming}, \textbf{Openssl}, \textbf{Opengl remotization}, insights on the \textbf{USB spec}, \textbf{Kernel module programming}}

\section{Employment history}
\cventry{2011--Present}{Developer consultant}{NICE s.r.l}{}{}{Develop NICE Desktop Cloud Visualization (DCV), an advanced technology that enables Technical Computing users to \textbf{remote access 2D/3D interactive applications} over a standard network.}

\section{Voluntary work}
\cventry{2008--Present}{Developer and maintainer}{GNOME Foundation}{}{}{Develop new features, maintain and bug fixing of the projects: gedit, gtksourceview, gitg, libgit2-glib. Also, contributed to main libraries like GTK+, glib, libgit2 and contributed to most of the GNOME modules. Information of my contributions can be found on the git repositories.}
\cventry{2011--Present}{RPM packager}{Fedora Project}{}{}{\textbf{Create} and \textbf{maintain RPM packages} for several free software applications including gedit, gtksourceview, gitg, libgit2 or libgit2-glib}
\cventry{2013}{Google Summer of Code}{Google}{}{}{\textbf{Mentor} university students to improve gedit, the default text editor of GNOME.}
\cventry{2007--2012}{Gtranslator developer and maintainer}{GNOME}{}{}{Gtranslator is an enhanced gettext PO file editor for the GNOME
desktop environment. During this time I developed new features, maintained and provided lots of bug fixes.}
\cventry{2010}{Google Code-in}{Google}{}{}{\textbf{Mentor} high school students to make small tasks on free software programs.}
\cventry{2004--2010}{Galician translation coordinator}{GNOME}{}{}{Translate and coordinate the translation of the GNOME applications for the Galician language.}

\section{Complementary training}
\cventry{2010}{Google Summer of Code}{Google}{}{}{Work made to \textbf{develop new features} for gedit Text Editor and \textbf{GTK+}.}
\cventry{2008}{Practical training}{Igalia}{Pontevedra}{}{Work related with free software, focused on GNOME technologies.}

\section{Education}
\cventry{2009--2012}{Computer engineer}{University of Vigo}{Ourense}{}{Extension of the previous degree. Both degrees are in European terms like a grade and a one year master.}
\cventry{2010--2011}{Erasmus}{Akademia Techniczno-Humanistyczna}{Bielsko-Biała}{}{}  % arguments 3 to 6 can be left empty
\cventry{2004--2008}{Technical engineer in computer management}{University of Vigo}{Ourense}{}{}

\section{Languages}
\cvlanguage{Spanish}{Mother tongue}{}
\cvlanguage{Galician}{Mother tongue}{}
\cvlanguage{English}{Fluent}{}
\cvlanguage{Polish}{Elementary proficiency}{}
\cvlanguage{Italian}{Elementary proficiency}{}

\section{Computer skills}

\subsection{Programming and scripting languages}
\cvline{}{C, C\texttt{++}, Python, XML, Vala, Bash and \LaTeX}
\cvline{}{Web technologies, Javascript, Powershell, HTML, PHP, CSS, Java, C\texttt{\#}}

\subsection{Operating systems}
\cvline{}{GNU/Linux, Windows and Mac OSX}

\subsection{Linux distributions}
\cvline{}{Fedora, CentOS, Ubuntu, Debian and OpenSuse}

\subsection{Libraries}
\cvline{}{GTK+, Glib, Pango, libxml2, Cairo, Openssl}
\cvline{}{Clutter, MX and GStreamer}

\subsection{Version control systems}
\cvline{}{git, svn, cvs and bazaar}

\section{Personal attributes}
I consider myself to be a hardworking, punctual, sociable individual, desirous for learning new things.

\section{Interests}
\cvlistdoubleitem{Football}{Squash}
\cvlistdoubleitem{Travelling}{Climbing}
\cvlistdoubleitem{Reading}{Free Software}
\cvlistdoubleitem{Socialising}{Cycling}

\renewcommand{\listitemsymbol}{} % change the symbol for lists

\section{References}
\cvlistdoubleitem{\textbf{José Manuel Pintor Freire}}{\textbf{Paolo Borelli}}
\cvlistdoubleitem{Computer engineer at CERN}{Computer engineer at NICE}
\cvlistdoubleitem{jmpintorfreire@gmail.com}{paolo.borelli@gmail.com}

\end{document}
